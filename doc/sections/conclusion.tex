\section{Conclusion}
In this analysis, I kept expanding my knowledge in \sv while focusing on mastering the \uvm framework to structure reusable, scalable, and efficient testbenches for the functional verification of digital designs. Compared to the development process of the previous introductory step, the adoption of the \uvm resulted in faster and higher quality development. I was able to take advantage of an extensively tested code base, directing the time and effort saved towards the most challenging problems of the designs I had to verify.

The hands-on experience encompassed the verification of two diverse designs: Intel's Pentium IV adder and a fixed-size windowed register file. Especially with the increased complexity of the latter design, both \sv and the \uvm truly shined. Reflecting on the testbench I had developed in the \emph{Microelectronic Systems} course laboratories, I once again recognize the limitations of the directed testing approach. The test program consisted of \vhdl procedures implementing the recursive solution to the \emph{Tower of Hanoi} puzzle at the signal level. It proved to be very error-prone to develop and it only targeted a subset of the \dut features. In contrast, leveraging the high-level features of \sv within the \uvm framework, I could simplify the modeling of the memory management unit, exercise the \dut with constrained-random stimuli, and compare its responses with those of a behavioral reference model. All of this, while tracking progress through both code and functional coverage.

In summary, this endeavor underscores the paramount importance of acquiring expertise in advanced verification methodologies. The \sv language and the \uvm combined confer invaluable skills to tackle the challenges of complex digital designs.