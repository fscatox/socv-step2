% -------------- %
% ---- INFO ---- %
% -------------- %

% File: macros.tex

% This file contains some useful shortcuts to print long names or frequently
% used acronyms. Because we're lazy. That's it. 

% ---------------- %
% ---- MACROS ---- %
% ---------------- %

% Text macros
\newcommand{\dut}{\ac{dut}\xspace}
\newcommand{\alu}{\ac{alu}\xspace}
\newcommand{\sv}{SystemVerilog\xspace}
\newcommand{\uvm}{\ac{uvm}\xspace}
\newcommand{\vhdl}{VHDL\xspace}
\def\CC{{C\nolinebreak[4]\hspace{-.05em}\raisebox{.4ex}{\tiny\bf ++}}\xspace}
\newcommand{\questa}{\texttt{QueastaSim}\xspace}

% Units
\DeclareSIUnit \bit {bit}

% Listings
\setminted[]{
  frame=lines,
  linenos=false,
  framesep=2mm,
  baselinestretch=1.2,
  %bgcolor=mintedbackground,
  %fontsize=\scriptsize,
  samepage=false,
  obeytabs=true,
  tabsize=2
}
\newcommand{\svinline}[1]{\mintinline{systemverilog}{#1}}
\newcommand{\vhdlinline}[1]{\mintinline{vhdl}{#1}}

% overlay arrows
\usetikzlibrary{tikzmark,calc}
\newcommand{\VerticalShiftStartArrows}{0,-.05}%
\newcommand{\VerticalShiftEndArrows}{0,+.15}%
\newcommand{\DrawArrows}[3][]{%
    \coordinate (Start Mid) at ($(pic cs:#2Left)!0.5!(pic cs:#2Right) + (\VerticalShiftStartArrows)$);
    \coordinate (End Mid)   at ($(pic cs:#3Left)!0.5!(pic cs:#3Right) + (\VerticalShiftEndArrows)$);
    \draw[->, #1, thick] (Start Mid) to (End Mid);
}